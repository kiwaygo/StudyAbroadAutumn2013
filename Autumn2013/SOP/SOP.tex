\documentclass[a4paper, 12pt]{article}
\usepackage[utf8]{inputenc}
\usepackage[english]{babel}
\usepackage[left=0.8in,right=0.8in,top=0.8in,bottom=0.8in]{geometry}

\newcommand{\HRule}{\rule{\linewidth}{0.2mm}}
\newcommand{\Hrule}{\rule{\linewidth}{0.3mm}}


\makeatletter% since there's an at-sign (@) in the command name
\renewcommand{\maketitle}{
  \parindent=0pt% don't indent paragraphs in the title block
  \begin{flushleft}
  \bf \large{\@author}
  \HRule
  \end{flushleft}
  \begin{center}
    \MakeUppercase{\bf \@title}
  \end{center}%
    \par
}
\makeatother% resets the meaning of the at-sign (@)

\title{Statement of Purpose} 
\author{Chi-Wei Tseng}
\date{Date: \today}

\linespread{1.2}

\begin{document}
{\linespread{0.8} \maketitle}
\parindent 5ex
I am aspired to a career as an artist more than a scientist.\\

From the viewpoint of not only appreciation, but creation, forms of digital art --- graphics, animation and music --- have long fascinated me with their high expressibility and power in storytelling since my childhood. As a kid who gained much sense of achievement by exhibiting his works to friends, I partcipated in virtually every extracurricular projects in designing class website, obtaining valuable experiences in digital image editing and photography. Later, amazed at the features produced by animation studios like Pixar and Dreamworks, I took a step further to self-study the creation of three-dimensional computer graphics(3DCG) using commercial software packages, and led a team to give introductory lectures about 3DCG at the information club of Taipei Municiple Chien-Kuo Senior High School. It was through the course of active exploration that I realise the strengths and weaknesses of computers as production tools for artists, and understand the neccessity to narrow the gap between technology and people before the tools can provide controls of higher degrees of freedom, as well as become more personalized and expressive. Only when artists are allowed to form their own tool, the power of computation can be handed over to them effectively.

% + With this in mind, 我以此為志向, 大學進入台大資訊系, 並希望在 xx 學校資訊系作 graduate study

% hwang fu: a gap between high school experience and tool development, no need to tell high school name.



\end{document}
%for the Ph.D. program. 
%
%My theoretical background can certainly provide me special insights in experiment designs, and with skillful handicrafts, I am confident about my ability in manipulation of experimental settings.  These traits make me special
%, just like the invention of 2D-electronic spectroscopy in research of photosynthesis
%Multidisciplinary researches are thriving, and there are more and more studies across the fields of biology, physics, and chemistry. Take the birds for example, structural coloration in bird feathers has been studies by scientists from physics and material engineering fields in the recent decade. Evolutionary studies are now employing chemical analysis technique to figure out the color of fossil feathers. However, I see so many questions about birds that could be answered by a physical chemists, but very few of them have been really studied in chemistry field. The studies in structural coloring carried out by material engineers are mostly from a top-down point of view, focusing on nano-structures of the barbs, but seldom from a bottom-up angle, discussing the effects of molecular properties and arrangements. Even more, how does the strut structures grow in a bird's bone? How birds adapt their four-color visions in the molecular level?  These questions may have been studied by biologists, but from the angle of physical chemistry, we surely can give some extraordinary insights.
% 
%
%I am passionate in birds. I recognize virtually all species of the birds in the campus. In my freshman year, I visited a bird nest almost every day to track the maturing of the chicks until they leave the nest. I took an ornithology course in my senior year and received the highest grade even I already have enough credits to graduate.
%
% We expect the method to be widely utilized in systems of photosynthetic complexes, as well as in organic solar energy materials.
% 
%I am also passionated in solving the puzzles of circadian clock. Nowaday explanations to circadian clocks are based on a complicated network of feedback loops between various chemicals and protein receptors. The experiments in this field are usually conducted through standard biochemistry procedures and the physical models are mostly built with series of rate equations to fit experimental results. However, I believe there must be more fundamental and more universal physics behind biological clock. 

% I have realized the importance of experimental works to theoretical studies from my research experiences. Although theoretical approaches are powerful tools for us to understand mechanisms behind natural phenomena, it is easy to lost directions and meanings without solid experimental evidences. 

