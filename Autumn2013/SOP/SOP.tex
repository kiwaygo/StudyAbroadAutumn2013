\documentclass[a4paper, 12pt]{article}
\usepackage[utf8]{inputenc}
\usepackage[english]{babel}
\usepackage[left=0.8in,right=0.8in,top=0.8in,bottom=0.8in]{geometry}

\newcommand{\HRule}{\rule{\linewidth}{0.2mm}}
\newcommand{\Hrule}{\rule{\linewidth}{0.3mm}}


\makeatletter% since there's an at-sign (@) in the command name
\renewcommand{\maketitle}{
  \parindent=0pt% don't indent paragraphs in the title block
  \begin{flushleft}
  \bf \large{\@author}
  \HRule
  \end{flushleft}
  \begin{center}
    \MakeUppercase{\bf \@title}
  \end{center}%
    \par
}
\makeatother% resets the meaning of the at-sign (@)

\title{Statement of Purpose} 
\author{Chi-Wei Tseng}
\date{Date: \today}

\linespread{1.2}

\begin{document}
{\linespread{0.8} \maketitle}
\parindent 5ex

From the viewpoint of not only appreciation, but creation, digital art --- graphics, animation and music --- have long fascinated me with their high capability and expressibility. Since my childhood, I have gained much sense of achievement by exhibiting my works to friends. I worked on many extracurricular projects in website designing from which I earned valuable experiences in digital image editing. As I got older, amazed at the features produced by animation studios like Pixar and Dreamworks, I self-studied generating three-dimensional computer graphics (3DCG) with commercial softwares, and led a team to give introductory lectures about 3DCG at the information club of my high school. These were also the years where I had fun toying with midi keyboards and audio editing softwares, trying to compose my own pieces of music. Indeed, I have become addicted to expressing myself via various forms of digital art. Through my course of broad and vigorous exploration, I realized that the strength of computers as tools for artistic productions lies in its ability to generate complicated effects with sophisticated simulations, and yet its weakness lies in the difficulty in translating artistic intentions into programs and discrete sets of parameters. Hence, I aimed at narrowing the gap between technology and people so that digital artists can concentrate more on their works rather than their skills, and can effectively unleash full power of computation on artistic purposes.\\

% + With this in mind, 我以此為志向, 大學進入台大資訊系, 並希望在 xx 學校資訊系作 graduate study
% hwang fu: a gap between high school experience and tool development, no need to tell high school name.

My passion for digital art, especially for 3DCG, motivated me to dive head-first into a series of challenging courses in the university curriculum. These include ``Digital Image Synthesis'', ``Digital Visual Effects'', ``Computational Photography'' (lectured by Professor Yung-Yu Chuang), ``Interactive Computer Graphics'' (lectured by Professor Ming Ouyang), ``Digital Image Processing'' (lectured by Professor Ming-Sui Lee) and ``GPU Programming'' (lectured by Professor Wei-Chao Chen). I worked hard, particularly in the aspect of term projects, where I and other members in my team had great times hunting for interesting research papers about computer graphics, implementing the algorithms and adding our own touch to them to improve the result. In addition to the term projects, I and my friends have tried to wrote our own ray-tracer from scratch in the sophomore year. Although our knowledge in either rendering or coding were rather limited back then, we spent a month digging into books and gradually worked up the renderer. It was an unforgettable project, because of not what we have achieved but of the excitement from revealing the secrets behind 3DCG for the first time. (...)\\

% The lighting-by-guide project

(The lighting-by-guide project.)\\

% EC2D project:

As I originally enroll in university to study chemistry, a compromise for not doing well enough on the college entrance exam, I strived in my freshman year and got qualified in double-majoring computer science. While creative innovation and careful system analysis are the backbone of engineering, the ability to think critically and accurately is strongly emphasized in the field of natural science like chemistry. I got the best of both worlds. I joined the theoretical chemsitry lab led by Professor Yuan-Chung Cheng, where we verified novel designs of ultrafast nonlinear electronic spectroscopy by simulating quantum dynamics of molecules interacting with light. The work experience at the junction point of computer science and chemistry enables me to formulate and implement physically accurate models on computers. (...)\\
 
 
% Aspiration: artist career. To tell great stories. To convey my love toward great things in life. Yet, ...

Personally, I am aspired to a career as an artist more than a scientist or engineer. I enjoyed creating and telling stories, through which I conveyed my love toward great things in life. I collected preliminary ideas on my blog and gradually developed them into complete works. In my senior year, I voluntarily participated the graduate musical of the chemistry department as the director, where I led a brilliant team to write the scripts, to compose the scores, to design the stage and to act. Yet(?), my ultimate goal is to establish an animation studio in my homeland, Taiwan, producing movies as well as developing novel techniques in the production pipeline. \\


% I aim at 

I aim at (doing something) in my graduate studies.\\

\end{document}

% Hwang Fu's SOP for the Ph.D. program. 

%
%My theoretical background can certainly provide me special insights in experiment designs, and with skillful handicrafts, I am confident about my ability in manipulation of experimental settings.  These traits make me special
%, just like the invention of 2D-electronic spectroscopy in research of photosynthesis
%Multidisciplinary researches are thriving, and there are more and more studies across the fields of biology, physics, and chemistry. Take the birds for example, structural coloration in bird feathers has been studies by scientists from physics and material engineering fields in the recent decade. Evolutionary studies are now employing chemical analysis technique to figure out the color of fossil feathers. However, I see so many questions about birds that could be answered by a physical chemists, but very few of them have been really studied in chemistry field. The studies in structural coloring carried out by material engineers are mostly from a top-down point of view, focusing on nano-structures of the barbs, but seldom from a bottom-up angle, discussing the effects of molecular properties and arrangements. Even more, how does the strut structures grow in a bird's bone? How birds adapt their four-color visions in the molecular level?  These questions may have been studied by biologists, but from the angle of physical chemistry, we surely can give some extraordinary insights.
% 
%
%I am passionate in birds. I recognize virtually all species of the birds in the campus. In my freshman year, I visited a bird nest almost every day to track the maturing of the chicks until they leave the nest. I took an ornithology course in my senior year and received the highest grade even I already have enough credits to graduate.
%
% We expect the method to be widely utilized in systems of photosynthetic complexes, as well as in organic solar energy materials.
% 
%I am also passionated in solving the puzzles of circadian clock. Nowaday explanations to circadian clocks are based on a complicated network of feedback loops between various chemicals and protein receptors. The experiments in this field are usually conducted through standard biochemistry procedures and the physical models are mostly built with series of rate equations to fit experimental results. However, I believe there must be more fundamental and more universal physics behind biological clock. 

% I have realized the importance of experimental works to theoretical studies from my research experiences. Although theoretical approaches are powerful tools for us to understand mechanisms behind natural phenomena, it is easy to lost directions and meanings without solid experimental evidences. 

