\documentclass[a4paper, 12pt]{article}
\usepackage[utf8]{inputenc}
\usepackage[english]{babel}
\usepackage[left=0.8in,right=0.8in,top=0.8in,bottom=0.8in]{geometry}

\newcommand{\HRule}{\rule{\linewidth}{0.2mm}}
\newcommand{\Hrule}{\rule{\linewidth}{0.3mm}}


\makeatletter% since there's an at-sign (@) in the command name
\renewcommand{\maketitle}{
  \parindent=0pt% don't indent paragraphs in the title block
  \begin{flushleft}
  \bf \large{\@author}
  \HRule
  \end{flushleft}
  \begin{center}
    \MakeUppercase{\bf \@title}
  \end{center}%
    \par
}
\makeatother% resets the meaning of the at-sign (@)

\title{Statement of Purpose} 
\author{Chi-Wei Tseng}
\date{Date: \today}

\linespread{1.2}

\begin{document}
{\linespread{0.8} \maketitle}
\parindent 5ex

From the viewpoint of not only appreciation, but creation, digital art --- graphics, animation and music --- have long fascinated me with its capability and expressibility. Since my childhood, I have gained much sense of achievement by exhibiting my works to friends. I worked on many extracurricular projects in website designing from which I earned practical experiences in digital image editing. As I got older, amazed at the features produced by animation studios like Pixar and Dreamworks, I self-studied generating three-dimensional computer graphics (3DCG) with commercial softwares, and gave introductory lectures about 3DCG at the information club of my high school. I also had fun toying with midi keyboards and audio editing softwares, trying to compose my own pieces of music. Indeed, I have become addicted to expressing myself via forms of digital art. Through broad and vigorous exploration, I realized that the strength of computers as tools for artistic productions lies in its ability to generate complicated effects and styles through simulation, and yet their weakness lies in the difficulty in translating artistic intentions into programs and parameters. Hence, I want to study toward a graduate degree in computer science, focusing on narrowing the gap between technology and people so that digital artists can concentrate more on their works rather than their skills, and thus effectively unleash full power of computation for artistic purposes.\\

% + With this in mind, 我以此為志向, 大學進入台大資訊系, 並希望在 xx 學校資訊系作 graduate study
% hwang fu: a gap between high school experience and tool development, no need to tell high school name.

My strong passion for digital art, especially for 3DCG, motivated me to dive head-first into a series of advanced courses in the university curriculum, equipping me with exceptional knowledge in related subjects. The courses included ``Digital Image Synthesis'', ``Digital Visual Effects'', ``Computational Photography'', ``Interactive Computer Graphics'', ``Digital Image Processing'' and ``GPU Programming.'' I worked with devotion, particularly in terms of projects, where I often led my teammates to hunt for inspiring research papers and contributed ways to modify, hybridize or combine algorithms in order to achieve our goal. I became confident in my leadership as my teammates commented that I had a good taste in selecting prominent topics (i.e.\ ``Milktea Rendering'' and `` Chord Arranger on A Field-Programmable Gate Array Board''), planned feasible roadmaps and provided encouraging goals and visions. Besides curricular works, I and friends have tried to write a ray-tracer from scratch in our sophomore year. Although our knowledge in both rendering and coding were limited back then, we spent months surveying and gradually built up the renderer. The excitement for opening up the blackbox of 3DCG for the first time in our life was unforgettable. This, along with other projects and courses, provided me opportunities to find my strength in collaborative software development, and to explore the mechanisms behind functionalities of many production tools I have used. With enthusiasm, I grasped them and prepared myself with a specialization in digital art that differentiated me from other students.\\

% The lighting-by-guide project

During my senior year, I participated in developing the ``Lighting-by-guide System'' under instruction of Professor~Yung-Yu~Chuang. The goal of the research project was a system to guess automatically the quantity, positions, intensities and colors of lights in a 3D scene from an artistic depiction of the anticipated render, namely the lighting-guide. We faced challenges from a high-dimensional optimization problem for energy functions with considerable discontinuities. In addition to implementing a hierarical search-and-reduction procedure in the space of lighting properties, I fine-tuned the algorithm through proposing an adequate optimizing target under a ``most-noticeable-light-first'' approach, achieving faster convergence and lower perceptual difference between the lighting-guide and the image rendered under the guessed lighting parameters. The method is readily applicable to the production pipeline of 3D animation, since the output of the system offers a convenient starting point from which lighting artists only need to make minute adjustments. Under the supervision of Professor Chuang, I have been given the freedom to push the project toward any direction I wanted, which implicitly forced me to build up the essential traits for an independent researcher in computer graphics. I learned to discover hidden bottlenecks and defects through benchmarking and analysing intermediate products, to ask crucial questions that may lead to great improvements in the result, and to widely consult resources from manuals to journals. Through working on this project, I have developed strong research fundamentals.\\

% EC2D project:

All is not possible if it were not for a turning point in my freshman year. As I originally enrolled in university to study chemistry, a compromise for not doing well enough on the college entrance exam, I strived in my freshman year and got qualified in double-majoring computer science. While creative innovation and careful system analysis are the backbone of engineering, the ability to think critically and accurately is strongly emphasized in the field of natural science like chemistry. I got the best of both worlds. I joined the theoretical chemsitry lab led by Professor~Yuan-Chung~Cheng, where we verified novel designs of ultrafast nonlinear electronic spectroscopy by simulating quantum dynamics of molecules interacting with light. The work experience at the junction point of computer science and chemistry enables me to mathematically formulate and implement physically accurate models on computers. Additionally, in order to write programs for simulation and to process obtained data, I got myself familiar with various numerical methods and visualization tools, which radically benefitted my capability in carrying out research projects. Most important of all, (...)\\
 
% Aspiration: artist career. To tell great stories. To convey my love toward great things in life. Yet, ...

Personally, I am aspired to a career as an artist more than an engineer or a scientist. I enjoyed writing stories, through which I conveyed my attitude toward life, nature and love. I collected preliminary ideas on my blog and gradually developed them into complete works. I voluntarily directed the graduate musical in 2012 of the chemistry department, where I led a brilliant team to write the scripts, to compose the scores, to design the stage and to act. Yet, my fond of storytelling forms my ultimate resolution to establish an animation studio in my homeland, Taiwan, producing movies as well as developing novel techniques in the production pipeline. While lack of funding and human resources severely hinder traditional animation industry from blooming in Taiwan, the prosperous status of our information/technology industry sets a much more concrete foundation for producing digital animation. However, mere passion is not enough to grant the fulfillment of this goal. I need to learn more and to recieve further trainings in doing research for the purpose of mastering the discipline of computer graphics. This is the reason I apply for XXX program at XXX university.\\

% match

Through a careful survey, I am convinced that your program suits me the best. (...)\\

\end{document}

% Hwang Fu's SOP for the Ph.D. program. 

%
%My theoretical background can certainly provide me special insights in experiment designs, and with skillful handicrafts, I am confident about my ability in manipulation of experimental settings.  These traits make me special
%, just like the invention of 2D-electronic spectroscopy in research of photosynthesis
%Multidisciplinary researches are thriving, and there are more and more studies across the fields of biology, physics, and chemistry. Take the birds for example, structural coloration in bird feathers has been studies by scientists from physics and material engineering fields in the recent decade. Evolutionary studies are now employing chemical analysis technique to figure out the color of fossil feathers. However, I see so many questions about birds that could be answered by a physical chemists, but very few of them have been really studied in chemistry field. The studies in structural coloring carried out by material engineers are mostly from a top-down point of view, focusing on nano-structures of the barbs, but seldom from a bottom-up angle, discussing the effects of molecular properties and arrangements. Even more, how does the strut structures grow in a bird's bone? How birds adapt their four-color visions in the molecular level?  These questions may have been studied by biologists, but from the angle of physical chemistry, we surely can give some extraordinary insights.
% 
%
%I am passionate in birds. I recognize virtually all species of the birds in the campus. In my freshman year, I visited a bird nest almost every day to track the maturing of the chicks until they leave the nest. I took an ornithology course in my senior year and received the highest grade even I already have enough credits to graduate.
%
% We expect the method to be widely utilized in systems of photosynthetic complexes, as well as in organic solar energy materials.
% 
%I am also passionated in solving the puzzles of circadian clock. Nowaday explanations to circadian clocks are based on a complicated network of feedback loops between various chemicals and protein receptors. The experiments in this field are usually conducted through standard biochemistry procedures and the physical models are mostly built with series of rate equations to fit experimental results. However, I believe there must be more fundamental and more universal physics behind biological clock. 

% I have realized the importance of experimental works to theoretical studies from my research experiences. Although theoretical approaches are powerful tools for us to understand mechanisms behind natural phenomena, it is easy to lost directions and meanings without solid experimental evidences. 

