\documentclass[a4paper, 11pt]{article}
\usepackage[utf8]{inputenc}
\usepackage[english]{babel}
\usepackage[left=0.8in,right=0.8in,top=0.8in,bottom=0.8in]{geometry}

\parskip=0.2in

\newcommand{\HRule}{\rule{\linewidth}{0.2mm}}
\newcommand{\Hrule}{\rule{\linewidth}{0.3mm}}


\makeatletter% since there's an at-sign (@) in the command name
\renewcommand{\maketitle}{
  \parindent=0pt% don't indent paragraphs in the title block
  \begin{flushleft}
  \bf \large{\@author} \hfill {\small \@date}
  \HRule
  \end{flushleft}
  \begin{center}
    \MakeUppercase{\bf \@title}
  \end{center}%
    \par
}
\makeatother% resets the meaning of the at-sign (@)

\title{Statement of Purpose} 
\author{Chi-Wei Tseng}
\date{Computer Science MS Program Applicant}

\linespread{1.2}

\begin{document}
{\large
{\linespread{0.8} \maketitle}
\parindent 5ex
}

Over the years, I have developed a keen desire in expressing myself through digital arts, computer graphics, animation, and music, whose capability and capacity as a vehicle of expression have fascinated me and satisfied my aspiration to create. I have designed numerous websites and graphical layouts, taught myself 3D computer graphics (3DCG), and played with midi keyboards and audio editing softwares. Although computers are powerful tools for rendering complicated effects and artistic styles through simulation, it remains difficult for artists to translate their intentions into input parameters or programming languages. Thus, I intend to pursue a graduate degree in compute graphics and contribute my talents to narrow the gap between technology and digital artistry so that the artists can effectively express their full creative potential.

%Passion for digital art, especially 3DCG, has ignited my motivation to dive into a series of advanced courses in university curriculum, which equipped me with profound understanding in image synthesis, image processing, digital visual effects, computational photography and real-time rendering. I worked diligently, particularly in projects, where I often led others to hunt for inspiring papers and brainstorm ideas to redesign algorithms for specific goals. Teammates have praised my contributions in proposing prominent topics (i.e.\ ``Milktea Rendering''), planning feasible roadmaps and providing encouraging visions. Apart from curricular works, friends and I took months to survey and write a ray tracer in sophomore year. The excitement for opening the blackbox of 3DCG for the first time was unforgettable. I grasped the opportunities provided by these projects and courses to strengthen my capabilities in collaborative software development and to explore mechanisms behind functionalities of many tools I once used. My specialized knowledge in digital art has differentiated me from other students.

My passion for the digital arts, especially 3DCG, has motivated me to take advanced courses in image synthesis, image processing, digital visual effects, computational photography, and real-time rendering. I have led many course projects including designing systems for art productions ranging diversely from a particle effect generator to a musical chord arranger. Notably, friends and I spent several months surveying and writing a ray tracer program accelerated with streaming SIMD extension in our sophomore year. This was my first encounter with what is inside the blackbox of 3DCG and the excitement was unforgettable. These collaborative software projects provided me an in-depth understanding of the technical basis of many art tools that I once used. Such hands-on experience with the engineering aspect of digital art uniquely qualifies me for the Program of Computer Graphics at Corenell.

%My passion for digital art, especially 3DCG, has motivated me to take a series of advanced courses in image synthesis, image processing, digital visual effects, computational photography and real-time rendering. I led several term projects in designing systems ranging from a particle effect generator to a musical chord arranger. Notably, friends and I took months to survey and write a SIMD accelerated ray tracer in our sophomore year. The excitement for opening the blackbox of 3DCG for the first time was unforgettable. These collaborative software projects provided me in-depth understanding of the technical basis of many tools for digital art I once used. Such hands-on experience uniquely qualifies me for the Computer Science Master Program in Real-World Computing.

% The lighting-by-guide project

I am most proud of my contribution to the development of ``Lighting-By-Guide System.'' As an undergraduate student of Communications and Multimedia Lab, I was in charge of the project under the supervision of \mbox{Professor~Yung-Yu~Chuang} in 2011 and 2012. Through this experience I acquired the essential traits that are needed of an independent researcher in computer graphics. I have learned to discover bottlenecks through benchmarking and examining intermediate products, consult resources from journals to manuals on encountering obstacles, and to ask crucial questions which can lead to significant improvements. I managed the project through all engineering phases in the development of the ``Lighting-By-Guide'' algorithm, which predicts the quantity, positions, and properties of lights in a 3D scene from an artistic depiction of the anticipated render. The system largely reduces workloads of lighting artists by offering them a lighting configuration which closely resembles the conceptual illustration. My proposal of the ``most-noticeable-light-first'' optimization target has succesfully achieved a fast convergence and low perceptual difference between lighting-guides and images rendered with the particularly predicted set of lights.
 
% EC2D project:

My proficiency and foundation in research came not only from my abundant engineering experiences, but also from the concrete scientific trainings that I had recieved. Since 2008, I have double-majored in chemistry and computer science. As a natural science major, the study of chemistry has instilled a methodical emphasis on thinking critically and logically, both of which have complemented my training in the engineering with its emphasis on innovation and systematic analysis. I was invited to join the Theoretical Chemistry Lab led by \mbox{Professor~Yuan-Chung~Cheng}. We verified innovative designs of nonlinear electronic spectroscopy by simulating the quantum dynamics of molecules interacting with laser pulses. Through the formulation and the implementation of physically accurate models on computers, I acquired experiences working with various numerical algorithms and visualization packages. Moreover, the environment has cultivated my down-to-earth attitude toward solving issues whose reasons are not understood. To me, revealing the causes and effects of phenomena is far more urgent than getting codes to run. These skill sets and reflection of a scientist are both transferable as well as have contributed to much of my success with the ``Lighting-By-Guide'' project. 

% All is not possible if it were not for a turning point in my freshman year. As I originally enrolled in university to study chemistry, a compromise for not doing well enough on the college entrance exam,  

% Aspiration: artist career. To tell great stories. To convey my love toward great things in life. Yet, ...

My ultimate goal is to leverage both my scientific and engineering background as well as my strong desire in creating digital artworks to establish an animation studio both in the U.S. and abroad, including Taiwan. I hoped to work with a wide range of projects including the production of movies and developing intuitive tool sets for individual digital animators. Currently, existing production pipelines for animated films are designed primarily for large commercialized studios. As such, they hardly suit the need for individual animators with smaller budget. I hope to help create a solution that will revolutionize the ways artists can convey their intentions onto their machines, so that they can fully share the artistic potential of computers and, thereby, liberating their imagination on what can be created or achieved. While mere passion is not enough to grant the fulfillment of this goal, I fully recognize the necessity in obtaining additional trainings as well as continuing further research to master the field of computer graphics at your prestigious institution.

The Computer Graphics MS Program at Cornell, directed by \mbox{Professor~Donald~Greenburg} known for Cornell Box and his contibution to rendering the effects of light diffusion between surfaces, with full access to a large array of graphics courses and sister disciplines, offers the challenges and opportunities I seek. This include the opportunity to work with faculties such as \mbox{Kavita Bala} whose work significantly accelerated light-transport algorithms for challenging scenes, and efficiently handled global illumination for scenes with light clusters, \mbox{Professor~Doug~James} whose researches on integrating phyical simulations of graphics and sound achieved high realism, and \mbox{Professor~Steve~Marschner} who developed innovative methods to generate the appearances of cloth and hair in synthetic images. The challenge of working alongside the best and brightest student body will also be invaluable. Hence, I full-heartedly anticipate in studying computer science at Cornell. With the guidance from your faculty, my diverse experiences and dedication to excel, I am confident in extending the frontiers of computer graphics as well as commencing my career as an industrial researcher in computer animation with a promising start.

% Due to my educational background and a strong desire to create, I aspire to become an artist who invents like an engineer and ponders like a scientist. Despite my early interest in digital art, I had a hobby of excerpting inspirations in life and developing them into novels, playscripts, and even musicals. My ultimate resolution is to establish an animation studio in my homeland, Taiwan, producing movies and developing expressive tools for digital animators. While lack of funding and human resources severely hinder our traditional animation industry from blooming, the prosperous status of our information/technology industry sets a more concrete foundation for producing computer animation. However, since currently existed production pipelines are designed mainly for highly commercialized studios, they hardly suit the need for individual animators and our low capital-intensive environment. A solution is to revolutionize the ways artists can convey their intentions to computers so that detailed division of labor becomes unnecessary. Evidently, mere passion is not enough to grant the fulfillment of this goal. I need to recieve further trainings in doing research and to master the field of computer graphics, which is the reason I apply for MS program in computer science at Stanford University.

%A scientist digests observations into theories, while an engineer employs theories to invent. My personal aspiration is to combine the both and become an artist, who invents from observations. Indeed, I was enthusiastic in collecting preliminary ideas and developing them into novels or playscripts. In 2012, I voluntarily directed the graduate musical for department of chemistry, where I led a team to work up everything from scratch with much sense of achievement. My fond of storytelling resulted in my ultimate resolution to establish an animation studio in my homeland, Taiwan, producing movies and developing novel tools for digital animators. While lack of funding and human resources severely hinder our traditional animation industry from blooming, the prosperous status of our information/technology industry sets a more concrete foundation for producing computer animation. However, since currently existed production pipelines are designed mainly for highly commercialized studios, they hardly suit the need for our low capital-intensive environment. A solution is to revolutionize the ways artists can convey their intentions to computers so that detailed division of labor becomes unnecessary. Evidently, mere passion is not enough to grant the fulfillment of this goal. I need to recieve further trainings in doing research and to master the field of computer graphics, which is the reason I apply for XXX program at XXX university.

% match

%Through a careful survey, I am convinced that your MS program in computer science can provide me with the best assistance in pursuing my ambition. I hope to learn from your distinguished faculty of many renowned professors who pioneer the field of computer graphics. I have heard of Professor~Ron~Fedkiw for his famous level-set method in graphical simulations, Professor~Vladlen~Koltun for his works in simplifying 3D content creation, and Professor~Pat~Hanrahan for his contribution to Pixar's Renderman and research in rendering subsurface scattering. Additionally, high academic capability of your students lead to an ambience which encourages self-advancement, that I long for. The ambience has even penetrated over the distance through Internet, as one of my projects, ``Milktea Rendering'', was actually inspired by the amazing renders your students generated in the rendering competition of the course ``Image Synthesis Techniques.'' Hence, I full-heartedly anticipate in studying computer science at Stanford. With the guidance from your faculty, my diverse experiences and dedication to excel, I am confidnet in extending the frontiers of computer graphics as well as commencing my career as a digital artist with a promising start.

\end{document}
